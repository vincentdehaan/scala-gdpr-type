\documentclass[aspectratio=169]{beamer}
\usepackage[T1]{fontenc}
\usepackage{beramono}
\usepackage{listings}
\usepackage{xcolor}

\definecolor{dkgreen}{rgb}{0,0.6,0}
\definecolor{gray}{rgb}{0.5,0.5,0.5}
\definecolor{mauve}{rgb}{0.58,0,0.82}

\lstdefinestyle{myScalastyle}{
  frame=tb,
  language=scala,
  aboveskip=2mm,
  belowskip=2mm,
  showstringspaces=false,
  columns=flexible,
  basicstyle={\tiny\ttfamily},
  numbers=none,
  numberstyle=\tiny\color{gray},
  keywordstyle=\color{blue},
  commentstyle=\color{dkgreen},
  stringstyle=\color{mauve},
  frame=single,
  breaklines=true,
  breakatwhitespace=true,
  tabsize=3,
}
\beamertemplatenavigationsymbolsempty

\usepackage[english]{babel}


\title{Enhance GDPR compliance using type level programming}
\author{Vincent de Haan}
\date{}

\begin{document}

\frame{\maketitle}

\frame[plain]{
  \frametitle{Who am I?}

  \begin{itemize}
    \item Self-employed Scala engineer $\ldots$
    \item $\dots$ in possession of a law degree.
  \end{itemize}
}

\frame{
  \frametitle{Goals for today}

  \begin{itemize}
  \item Learn something about \textit{type level programming}, \textit{macros} and \textit{monads}.
  \item Apply this knowledge to a real business problem.
  \end{itemize}
}

\frame{
  \frametitle{Art. 30, sect 1 of the General Data Protection Regulation (GDPR)}

  \begin{quote}
  Each controller [...] shall maintain a \textbf{record of processing activities} under its responsibility. That record shall contain all of the following information:

[...]

(b) the purposes of the processing;

(c) [...] the categories of personal data;

(d) the categories of recipients to whom the personal data have been or will be disclosed [...];
  \end{quote}
}

\frame{
  \frametitle{The problem}
We need to document all our \textbf{processing activities}, but $\ldots$
  \begin{itemize}
  \item $\ldots$ documentation is always incomplete,
  \item $\ldots$ documentation is never up-to-date,
  \item $\ldots$ we have "hidden" features (data export, logging, bugs, admin tools) that are not documented.
  \end{itemize}
  
}

\frame{
  \frametitle{A crazy idea}

  Can we extract the documentation from the code?
}

\begin{frame}[fragile]
  \frametitle{The original idea}

Given this code: 
\begin{columns}
\begin{column}{0.3\textwidth}

\begin{lstlisting}[style=myScalastyle,frame=none]
trait PatientRepository {
  @Processing
  def getById(id: String):
    Future[Patient]
}
\end{lstlisting}


\end{column}

\begin{column}{0.7\textwidth}  

\begin{lstlisting}[style=myScalastyle,frame=none]
class PatientApi(patientRepo: PatientRepository) {
   val thePatient = (patientRepo.getById("someId")
     : @ProcessingInstance("Justification for the processing"))
}
\end{lstlisting}

\end{column}
\end{columns}

Create a compiler plugin that applies two rules:
\begin{itemize}
  \item For each method with a \texttt{@Processing} annotation, we need a \texttt{@ProcessingInstance} annotation at the call site.
  \item When we find a \texttt{@ProcessingInstance} annotation during compilation, report the purpose.
\end{itemize}

\end{frame}

\begin{frame}
  \frametitle{The problem with this approach}
  
  \begin{itemize}
    \item Programs have many layers (services, utils) between API and repository
    \pause
    \item Still interested in how to write a compiler plugin? Watch my previous talk online: https://www.youtube.com/watch?v=BZcPGHQlcAE
  \end{itemize}
\end{frame}


\frame{
  \frametitle{New idea: a \texttt{ProtectedData[\_]} monad } 

  \begin{itemize}
  \item Keep the protected data inside the \texttt{ProtectedData[\_]}
  \item Unwrap it at the and, and report it's purpose
  \item Record the \textit{original} data type in a second type parameter
  \end{itemize}

}

\begin{frame}[fragile]
\frametitle{The \texttt{ProtectedData} class}

\begin{lstlisting}[style=myScalastyle,frame=none]
class ProtectedData[T, H <: HList] private[scalagdpr](private[scalagdpr] val value: T) {

  def get(justification: DataProcessingJustification[H]): T = value
}
\end{lstlisting}
\pause
\begin{lstlisting}[style=myScalastyle,frame=none]
class DataProcessingJustification[H <: HList] private[scalagdpr]

object DataProcessingJustification {

  def apply[Purpose, Subjects, Recipients] = new Object {
    def apply[H <: HList](implicit r: DataProcessingReporter[Purpose, Subjects, Recipients, H]) =
      new DataProcessingJustification[H]
  }
}
\end{lstlisting}

\end{frame}

\begin{frame}[fragile]
\frametitle{The \texttt{DataProcessingReporter}}

\begin{lstlisting}[style=myScalastyle,frame=none]
class DataProcessingReporter[Purpose, Subjects, Recipients, H] {}

object DataProcessingReporter {
  implicit def reporter[Purpose, Subjects, Recipients, H <: HList]: DataProcessingReporter[Purpose, Subjects, Recipients, H] =
  macro report_impl[Purpose, Subjects, Recipients, H]
\end{lstlisting}
\pause
\begin{lstlisting}[style=myScalastyle,frame=none]
  def report_impl[Purpose, Subjects, Recipients, H](c: Context): c.Expr[DataProcessingReporter[Purpose, Subjects, Recipients, H]] = {
    import c.universe._
    val TypeApply(_, purposeTree :: subjectsTree :: recipientsTree :: hTree :: Nil) = c.macroApplication

    val dataTypes = hTree.toString.split(" :: ").dropRight(1).map(dt => if(dt.contains("\"")) dt.split("\"").tail.head else dt)

    val report =
      s"""
         |>>> Data processing:
         |Purpose: ${purposeTree.toString()}
         |Subjects: ${subjectsTree.toString()}
         |Recipients: ${recipientsTree.toString()}
         |Data types: ${dataTypes.mkString(", ")}
         |""".stripMargin

    println(report)
    c.Expr(q"new nl.vindh.scalagdpr.DataProcessingReporter[$purposeTree, $subjectsTree, $recipientsTree, $hTree]")
  }
\end{lstlisting}

\end{frame}


\begin{frame}[fragile]
\frametitle{Overview}

\begin{enumerate}
\item
\begin{lstlisting}[style=myScalastyle,frame=none]
val protectedData: ProtectedData[Int, T]
protectedData.get(???)
\end{lstlisting}
We need a \footnotesize{\texttt{DataProcessingJustification[T]}}
\pause
\item
\begin{lstlisting}[style=myScalastyle,frame=none]
ProtectedDataJustification["My purpose", "Some subjects", "And recipients"]
\end{lstlisting}
We define the purpose using \textit{literal types}.
\pause
\item
\begin{lstlisting}[style=myScalastyle,frame=none]
ProtectedDataJustification["My purpose", "Some subjects", "And recipients"].apply
\end{lstlisting}
\footnotesize{\texttt{T}} is inferred to create a \footnotesize{\texttt{DataProcessingJustification[T]}}.
\end{enumerate}

\end{frame}

\begin{frame}[fragile]
\frametitle{Overview II}

\begin{enumerate}
\item[4.]
\begin{lstlisting}[style=myScalastyle,frame=none]
ProtectedDataJustification["My purpose", "Some subjects", "And recipients"].apply(implicit r: DataProcessingReporter[Purpose, Subjects, Recipients, H])
\end{lstlisting}
We need to search for an implicit \texttt{DataProcessingReporter} with the right type parameters.
\item[5.]

\begin{lstlisting}[style=myScalastyle,frame=none]
  implicit def reporter[Purpose, Subjects, Recipients, H <: HList]: DataProcessingReporter[Purpose, Subjects, Recipients, H] =
  macro report_impl[Purpose, Subjects, Recipients, H]
\end{lstlisting}
This code runs the macro and creates a \texttt{DataProcessingReporter}.
\end{enumerate}
\end{frame}

\begin{frame}[fragile]
\frametitle{A simple repository}

\begin{columns}
\begin{column}{0.3\textwidth}

\begin{lstlisting}[style=myScalastyle,frame=none]
case class Patient(
  id: String,
  name: String,
  address: String)
  
case class MedicalRecord(
  id: String,
  patientId: String,
  disease: String,
  treatment: String)
\end{lstlisting}


\end{column}
\vrule
\begin{column}{0.7\textwidth}  

\begin{lstlisting}[style=myScalastyle,frame=none]
type ProtectedDataSource[T] = ProtectedData[T, shapeless.::[T, HNil]]  


trait PatientRepository {
  def getPatientById(id: String): ProtectedDataSource[Patient]
}

class PatientRepositoryImpl {
  def getPatientById(id: String): ProtectedDataSource[Patient] =
    ProtectedData {
      dbconnection.get(...)
    }
}

\end{lstlisting}


\end{column}
\end{columns}

\end{frame}

\begin{frame}[fragile]
\frametitle{Handling lists}
\begin{columns}
\begin{column}{0.2\textwidth}

\begin{lstlisting}[style=myScalastyle,frame=none]
case class Patient(
  id: String,
  name: String,
  address: String)
  
case class MedicalRecord(
  id: String,
  patientId: String,
  disease: String,
  treatment: String)
\end{lstlisting}


\end{column}
\pause

\begin{column}{0.8\textwidth}  

\begin{lstlisting}[style=myScalastyle,frame=none]
type ProtectedDataSource[T] = ProtectedData[T, shapeless.::[T, HNil]] 
type ProtectedDataSourceF[F[_], T] = ProtectedData[F[T], shapeless.::[T, HNil]]
 

trait PatientRepository {
  def getPatientById: ProtectedDataSource[Patient]
}

class PatientRepositoryImpl {
  def getPatientById: ProtectedDataSource[Patient] =
    ProtectedData {
      // get the patient
    }
    
  def getAllPatients: ProtectedDataSourceF[Seq, Patient] =
    ProtectedData {
       // get all patients as a Seq[Patient]
    }
}

\end{lstlisting}


\end{column}
\end{columns}

\end{frame}

\begin{frame}[fragile]
\frametitle{Transforming \texttt{ProtectedData}}

Given a \texttt{ProtectedData[Patient, \_]}, how to get a \texttt{ProtectedData} holding a \texttt{String} with the patient's name?
\pause
Let's define \texttt{map}:
\begin{lstlisting}[style=myScalastyle,frame=none]
def map[U](f: T => U): ProtectedData[U, H] =
  ProtectedData {
    f(value)
  }
\end{lstlisting}
\pause
Now we have:
\begin{lstlisting}[style=myScalastyle,frame=none]
val patient: ProtectedData[Patient, Patient :: HNil] = // some protected data
val name: ProtectedData[String, Patient::HNil] = patient.map(_.name)
\end{lstlisting}

\end{frame}

\begin{frame}[fragile]
\frametitle{How about \texttt{flatMap}?}

Given:

\begin{lstlisting}[style=myScalastyle,frame=none]
case class Patient(id: String, name: String, address: String)
case class MedicalRecord(id: String, patientId: String, disease: String, treatment: String)

def getPatientByName(name: String): ProtectedDataSource[Patient]
def getMedicalRecordByPatientId(patientId: String): ProtectedDataSourceF[Seq, MedicalRecord]
\end{lstlisting}

How to get all records of Mary?
\pause


\begin{lstlisting}[style=myScalastyle,frame=none]
val maryId: ProtectedData[String, Patient::HNil] = getPatientByName("Mary").map(_.patientId)

val maryRecords: ProtectedData[Seq[MedicalRecord], ???] = getMedicalRecordByPatientId(???)
\end{lstlisting}


\end{frame}

\begin{frame}[fragile]
\frametitle{Intermezzo on HLists}

\begin{itemize}
  \item HList = heterogenous list
  \item part of the Shapeless library
  \pause
  \item Can hold values of multiple types:
  \begin{lstlisting}[style=myScalastyle,frame=none]
val hlist: String :: Int :: HNil = "foo" :: 42 :: HNil
\end{lstlisting}
  \pause
  \item We have type classes for list operations
  
  \begin{lstlisting}[style=myScalastyle,frame=none]
  val hlist1: String :: Int :: HNil = "foo" :: 42 :: HNil
  val hlist2: Int :: HNil = 43 :: HNil

  val prepend = implicitly[Prepend[String :: Int :: HNil, Int :: HNil]]

  val newHList = prepend.apply(hlist1, hlist2)
  \end{lstlisting}
\end{itemize}

\end{frame}

\begin{frame}[fragile]
\frametitle{How about \texttt{flatMap}? II}
Notice the type of \texttt{val maryRecords}:
\begin{lstlisting}[style=myScalastyle,frame=none]
val maryId: ProtectedData[String, Patient::HNil] = getPatientByName("Mary").map(_.patientId)

val maryRecords: ProtectedData[Seq[MedicalRecord], MedicalRecord::Patient::HNil] = getMedicalRecordByPatientId(???)
\end{lstlisting}
\pause
Let's define \texttt{flatMap}:
\begin{lstlisting}[style=myScalastyle,frame=none]
class ProtectedData[T, H <: HList] private[scalagdpr](private[scalagdpr] val value: T) {
  def flatMap[U, H1 <: HList](f: T => ProtectedData[U, H1])(implicit p: Prepend[H, H1]): ProtectedData[U, p.Out] =
    f(value).asInstanceOf[ProtectedData[U, p.Out]]
\end{lstlisting}
\pause
Now we can write:
\begin{lstlisting}[style=myScalastyle,frame=none]
val records: ProtectedData[Seq[MedicalRecord], MedicalRecord::Patient::HNil] = for {
  maryId <- getPatientByName("Mary").map(_.patientId)
  maryRecords <- getMedicalRecordByPatientId(maryId)
} yield maryRecords
\end{lstlisting}

\end{frame}

\begin{frame}[fragile]
\frametitle{How about asynchronous programming?}
Given:
\begin{lstlisting}[style=myScalastyle,frame=none]
case class Patient(id: String, name: String, address: String)
case class MedicalRecord(id: String, patientId: String, disease: String, treatment: String)

def getPatientByName(name: String): Future[ProtectedDataSource[Patient]]
def getMedicalRecordByPatientId(patientId: String): Future[ProtectedDataSourceF[Seq, MedicalRecord]]
\end{lstlisting}
How to get all records of Mary?
\pause
\begin{lstlisting}[style=myScalastyle,frame=none]
val maryRecords: Future[ProtectedData[Seq[MedicalRecord], MedicalRecord::Patient::HNil]] = for {
  protectedPatient <- getPatientByName("Mary")
  protectedRecord <- ???
} yield ???
\end{lstlisting}

\end{frame}

\begin{frame}[fragile]
Let's create a monad transformer: \texttt{ProtectedDataT}

\begin{lstlisting}[style=myScalastyle,frame=none]
final case class ProtectedDataT[F[_], T, H <: HList](value: F[ProtectedData[T, H]]) {
  def map[U](f: T => U)(implicit F: Functor[F]): ProtectedDataT[F, U, H] =
    ProtectedDataT(F.map(value)(pd => pd.map(f)))

  def flatMap[U, H1 <: HList](f: T => ProtectedDataT[F, U, H1])(implicit p: Prepend[H, H1], F: Monad[F]): ProtectedDataT[F, U, p.Out] =
    flatMapF(a => f(a).value).asInstanceOf[ProtectedDataT[F, U, p.Out]]

  def flatMapF[U, H1 <: HList](f: T => F[ProtectedData[U, H1]])(implicit p: Prepend[H, H1], F: Monad[F]): ProtectedDataT[F, U, p.Out] =
    ProtectedDataT(F.flatMap(value)(getProtected andThen f)).asInstanceOf[ProtectedDataT[F, U, p.Out]]

  private val getProtected = (pd: ProtectedData[T, H]) => pd.get(DataProcessingJustification.unsafeJustification)

}
\end{lstlisting}

\end{frame}

\begin{frame}[fragile]
\frametitle{How about asynchronous programming? ||}
Given:
\begin{lstlisting}[style=myScalastyle,frame=none]
case class Patient(id: String, name: String, address: String)
case class MedicalRecord(id: String, patientId: String, disease: String, treatment: String)

def getPatientByName(name: String): Future[ProtectedDataSource[Patient]]
def getMedicalRecordByPatientId(patientId: String): Future[ProtectedDataSourceF[Seq, MedicalRecord]]
\end{lstlisting}
How to get all records of Mary?
\pause
\begin{lstlisting}[style=myScalastyle,frame=none]
val maryRecords: Future[ProtectedData[Seq[MedicalRecord], MedicalRecord::Patient::HNil]] = (for {
  patient <- ProtectedDataT(getPatientByName("Mary"))
  records <- ProtectedDataT(getMedicalRecordByPatientId(patient.id))
} yield records).value
\end{lstlisting}

\end{frame}

\begin{frame}[fragile]
\frametitle{A complete example}

\begin{lstlisting}[style=myScalastyle,frame=none]
class ReportService(personRepo: PersonRepo, medicalRecordRepo: MedicalRecordRepo) {
  def getReportByName(name: String): ProtectedDataT[Future, String, Person :: MedicalRecord :: HNil] =
    for {
      person <- personRepo.getPersonByName(name)
      records <- medicalRecordRepo.getMedicalRecordsByPersonId(person.id)
    } yield s"""Medical report
               |Name: ${person.name}
               |Address: ${person.address}
               |
               |Treatments: ${records.map(_.treatment).mkString(", ")}
               |""".stripMargin}


class ReportApi (reportService: ReportService) {
  val route = path("reports") {
    get {
      parameters('patientName.as[String]) { patientName =>
        complete {
          reportService.getReportByName(patientName).value.map {
            protectedReport => protectedReport.get(
              DataProcessingJustification
                .withPurpose["Generate the patient report"]
                .withSubjects["Patients"]
                .withRecipients["Other doctors"].apply
            )}}}}}}

\end{lstlisting}

\end{frame}

\begin{frame}
\frametitle{Questions?}


\end{frame}

\end{document}
